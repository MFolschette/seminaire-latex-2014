%%% Fichier de préambule avec toutes les définitions communes aux trois diaporamas

% Packages
\usepackage[utf8]{inputenc}  % Encodage : UTF-8
\usepackage[T1]{fontenc}     % Caractères français

\usepackage[french]{babel}   % Langue française

\usepackage{amsmath}         % Maths améliorées
\usepackage{amssymb}         % Symboles mathématiques
\usepackage{amsfonts}        % Polices mathématiques
\usepackage{graphicx}        % Inclusion d'images (PNG, GIF, PDF)
\usepackage{array}           % Tableaux améliorés
\usepackage[normalem]{ulem}  % Styles barré et souligné
\usepackage{listings, multicol, xcolor} % Listings (code source)
\usepackage{lmodern}         % Police d'écriture
\usepackage{url, hyperref}   % Liens hypertexte

\usepackage{tikz}            % TikZ : figures avancées
\usepackage{circuitikz}      % Circuits électriques en TikZ
% Diverses bibliothèques pour TikZ
\usetikzlibrary{chains, calc, decorations.pathmorphing, fadings, shadings, arrows, decorations.pathreplacing, shapes} 



% Commande \meta pour les listings
\newcommand{\meta}[1]{\ensuremath\langle\itshape#1\ensuremath\rangle}

% Paramètres des listings
\lstset{
upquote=false,
columns=flexible,
basicstyle=\ttfamily\scriptsize,
language={[LaTeX]TeX},
identifiersstyle=\color{green},
emphstyle=\color{blue},
keywordstyle=\color{blue},
directivestyle=\color{blue},
commentstyle=\color{gray},
    inputencoding=utf8,
literate={eacute}{\'e}1,    % Permet de faire des accents dans un listing avec £\'e£, £\`e£, £\`a£
    {eagrave}{\`e}1,
    {aagrave}{\`a}1,
    escapechar={£},
    moretexcs={meta},
morekeywords={              % Mots-clefs supplémentaires
  part,chapter,subsection,subsubsection,
  frontmatter,mainmatter,backmatter,
  tableofcontents,listoffigures,listoftables,titlepage,
  includegraphics,includepdf,
  dddot,ddddot,
  frametitle,framesubtitle,
  pause,only,uncover,
  usetheme,usecolortheme,
  institute,maketitle,
  usetikzlibrary,
  node,path,
  commande
}
}



% Thème Beamer
\usetheme{CambridgeUS}



% Auteurs et licence
\author[Folschette, Jubien, Tanguy]{Maxime \textsc{Folschette\up{1} } \and Anthony \textsc{Jubien\up{2}} \and Julien \textsc{Tanguy\up{3}} \\ 
{\scriptsize  \up{1} IRCCyN équipe MeForBio \\
\up{2} IRCCyN équipe Robotique et ONERA Toulouse \\
\up{3} IRCCyN équipe Systèmes Temps Réels \\
maxime.folschette, anthony.jubien, julien.tanguy @irccyn.ec-nantes.fr} }
\institute[AED]{Association des Étudiants en Doctorat de l'ECN (AED)  \\ 
Document sous licence Creative Commons BY 3.0 FR \\
http://creativecommons.org/licenses/by/3.0/fr/}



% Suppression des icônes de navigation
\setbeamertemplate{navigation symbols}{}

% Sommaire au début de chaque partie
\AtBeginPart{%
    \frame{\partpage}

    \begin{frame}
        \frametitle{Plan}
        \tableofcontents
    \end{frame}
}
