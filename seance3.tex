\documentclass{beamer}

%%% Fichier de préambule avec toutes les définitions communes aux trois diaporamas

% Packages
\usepackage[utf8]{inputenc}  % Encodage : UTF-8
\usepackage[T1]{fontenc}     % Caractères français

\usepackage[french]{babel}   % Langue française
\frenchbsetup{PartNameFull=false}  % Nécessaire pour des parties correctement nommées dans un Beamer en français

\usepackage{amsmath}         % Maths améliorées
\usepackage{amssymb}         % Symboles mathématiques
\usepackage{amsfonts}        % Polices mathématiques
\usepackage{graphicx}        % Inclusion d'images (PNG, GIF, PDF)
\usepackage{array}           % Tableaux améliorés
\usepackage[normalem]{ulem}  % Styles barré et souligné
\usepackage{listings, multicol, xcolor} % Listings (code source)
\usepackage{lmodern}         % Police d'écriture
\usepackage{url, hyperref}   % Liens hypertexte
\usepackage{calc}            % Pour la commance \widthof

\usepackage{tikz}            % TikZ : figures avancées
\usepackage{circuitikz}      % Circuits électriques en TikZ
% Diverses bibliothèques pour TikZ
\usetikzlibrary{chains, calc, decorations.pathmorphing, fadings, shadings, arrows, decorations.pathreplacing, shapes} 



% Commande \meta pour les listings
\newcommand{\meta}[1]{\ensuremath\langle\itshape#1\ensuremath\rangle}

% Paramètres des listings
\lstset{
upquote=false,
columns=flexible,
basicstyle=\ttfamily\scriptsize,
language={[LaTeX]TeX},
identifiersstyle=\color{green},
emphstyle=\color{blue},
keywordstyle=\color{blue},
directivestyle=\color{blue},
commentstyle=\color{gray},
    inputencoding=utf8,
literate={eacute}{\'e}1,    % Permet de faire des accents dans un listing avec £\'e£, £\`e£, £\`a£
    {eagrave}{\`e}1,
    {aagrave}{\`a}1,
    escapechar={£},
    moretexcs={meta},
morekeywords={              % Mots-clefs supplémentaires
  part,chapter,subsection,subsubsection,
  frontmatter,mainmatter,backmatter,
  tableofcontents,listoffigures,listoftables,titlepage,
  includegraphics,includepdf,
  dddot,ddddot,
  frametitle,framesubtitle,
  pause,only,uncover,
  usetheme,usecolortheme,
  institute,maketitle,
  usetikzlibrary,
  node,path,
  commande
}
}



% Thème Beamer
\usetheme{CambridgeUS}
\usecolortheme{orchid}



% Auteurs et licence
\author[Folschette, Jubien, Tanguy]{Maxime \textsc{Folschette\up{1} } \and Anthony \textsc{Jubien\up{2}} \and Julien \textsc{Tanguy\up{3}} \\ 
{\scriptsize  \up{1} IRCCyN équipe MeForBio \\
\up{2} IRCCyN équipe Robotique et ONERA Toulouse \\
\up{3} IRCCyN équipe Systèmes Temps Réel \\
maxime.folschette, anthony.jubien, julien.tanguy @irccyn.ec-nantes.fr} }
\institute[AED]{Association des Étudiants en Doctorat de l'ECN (AED)  \\ 
Document sous licence Creative Commons BY 3.0 FR \\
\url{http://creativecommons.org/licenses/by/3.0/fr/}}



% Suppression des icônes de navigation
\setbeamertemplate{navigation symbols}{}

% Sommaire au début de chaque partie
\AtBeginPart{%
    \frame{\partpage}

    \begin{frame}
        \frametitle{Plan}
        \tableofcontents
    \end{frame}
}

% Boîte d'exemple
%\definecolor{couleurexemple}{RGB}{200,200,200}
\colorlet{couleurexemple}{green!20}

\newcommand{\myexgenerique}[3]{%
\colorbox{#2}{\begin{minipage}{#3\textwidth}%
#1
\end{minipage}}}

\newcommand{\myexsize}[2]{\myexgenerique{#2}{couleurexemple}{#1}}
\newcommand{\myex}[1]{\myexsize{1}{#1}}
\newcommand{\myextrueinline}[1]{\colorbox{couleurexemple}{#1}}
\newcommand{\myexinline}[1]{\quad\myextrueinline{#1}}

% \newcommand{\myex}[1]{%
% \colorbox{green!20}{\begin{minipage}{.8\textwidth}%
% #1
% \end{minipage}}}


\title[Séminaire \LaTeX, séance 3]{Séminaire \LaTeX, séance 3: Utilisation avancée}
\date{mardi 25 février 2014}



% Exemples de macros
\newcommand{\ltsname}{Diagramme de Transisions \'Etiquet\'e}
\newcommand{\abs}[1]{\ensuremath\left|#1\right|}
\newcommand{\lts}[1][]{\ensuremath\left(Q^{#1}, q_0^{#1}, A_{#1}, \rightarrow_{#1}\right)}



\begin{document}

%%%%%%%%% SLIDE %%%%%%%%%%%%%%%%%%

\begin{frame}
    \titlepage
\end{frame}

%%%%%%%%% SLIDE %%%%%%%%%%%%%%%%%%

\begin{frame}{Points abordés durant cette troisième séance}
    \begin{itemize}
            \item Bibliographie, 
            \item commandes avancées, 
            \item inclusion de figures à l'aide de différents outils, 
            \item création d'un diaporama à l'aide de la classe Beamer, 
            \item vos questions !
        \end{itemize}
\end{frame}

%%%%%%%%%%%%%%%%%%%%%%%%%%%%%%
%%%%%%%%%%% PART%% %%%%%%%%%%%%%%
%%%%%%%%%%%%%%%%%%%%%%%%%%%%%%

\part{Bibliographie}

%%%%%%%%%%

\section{BibTeX}

%%%%%%%%%% SLIDE %%%%%%%%%%%%%%%%%%

\begin{frame}[fragile]
\frametitle{Présentation de BibTeX}
BibTeX est un outil de gestion de bibliographie.

\medskip
La \emph{base de données} bibliographique est placée dans un fichier extérieur (\lstinline?.bib?).

\medskip
On inclut la bibliographie dans le document par les commandes :
\begin{lstlisting}
\bibliographystyle{plain}
\bibliography{nom-biblio}
\end{lstlisting}
Il est possible d'inclure plusieurs fichiers \lstinline?.bib? :
\lstinline?\bibliography{biblio1,biblio2}?.

\medskip
On crée des références dans le document par la commande : \lstinline?\cite{...}?  \cite{latexcompanion}.
\end{frame}

%%%%%%%%%%

\section{Exemple}

%%%%%%%%% SLIDE %%%%%%%%%%%%%%%%%%

\begin{frame}[fragile]
\frametitle{Exercice}

Créer un nouveau fichier \lstinline?.bib? nommé \lstinline?biblio.bib? et contenant : 

\begin{lstlisting}
@article{greenwade93,
    author  = "Inconnu",
    title   = "Titre",
    year    = "1993",
    journal = "Nom du journal",
    volume  = "14",
    number  = "3",
    pages   = "342--351"
}
\end{lstlisting}

Et y faire référence dans votre document principal:

\begin{lstlisting}
....
\cite{greenwade93}
....
\bibliographystyle{plain} %ou style alpha
\bibliography{biblio}
\end{lstlisting}

\end{frame}


%%%%%%%%%%

\section{JabRef}

%%%%%%%%%% SLIDE %%%%%%%%%%%%%%%%%%

\begin{frame}
\frametitle{Outils de gestion de bibliographie}

La plupart des bases de données bibliographiques permettent d'exporter une entrée en BibTeX (Google Scholar inclus : Préférences Scholar,  Gestionnaire des bibliographies,  Afficher les liens permettant d'importer des citations dans BibTeX).

\medskip
Utiliser un outil de gestion de bibliographie est nécessaire:
\begin{itemize}
\item JabRef,
\item Mendeley,
\item Zotero
\end{itemize}
\end{frame}

%%%%%%%%% SLIDE %%%%%%%%%%%%%%%%%%

\begin{frame}{Jabref (mutli-plateforme)}

\vspace*{-0.5cm}
\begin{figure}
\centering
\includegraphics[width=9cm]{img/jabref}
\end{figure}

\vspace*{-0.5cm}
{\footnotesize Téléchargement: \url{http://jabref.sourceforge.net/}}

\end{frame}


%%%%%%%%%%%%%%%%%%%%%%%%%%%%%%
%%%%%%%%%%% PART%% %%%%%%%%%%%%%%
%%%%%%%%%%%%%%%%%%%%%%%%%%%%%%


\part{Commandes avancées}

\section{Commandes personnalisées}

\begin{frame}[fragile]
\frametitle{Créer ses propres commandes}
Pourquoi?
\begin{itemize}
  \item Réutilisation
  \item Simplification
\end{itemize}

Définition

\begin{lstlisting}
\newcommand{\ltsname}{Diagramme de Transisions £\'E£tiquet£\'e£}
\newcommand{\abs}[1]{\left|#1\right|}
\newcommand{\lts}[1][]{\left(Q^{#1},q_0^{#1},A_{#1},\rightarrow_{#1}\right)}
\end{lstlisting}
Utilisation
  \begin{itemize}
\item \lstinline?\ltsname? \myexinline{\ltsname}
\item \lstinline?\abs{\pi}? \myexinline{$\abs{\pi}$}
\item \lstinline?\lts? \myexinline{$\lts$}\\
  \lstinline?\lts[n]? \myexinline{$\lts[n]$}
\end{itemize}
Restrictions
\begin{itemize}
  \item Pas de chiffres
  \item Pas de @
\end{itemize}
\end{frame}



\begin{frame}[fragile]
  \frametitle{Définir des commandes simples}

\begin{lstlisting}
\newcommand{\£\meta{nomcommande}£}[£\meta{n}£]{£\meta{Corps de la commande}£}
\end{lstlisting}
À chaque fois que la commande créée est utilisée, elle est remplacée par son contenu.

\bigskip
Exemple :
\begin{lstlisting}
\newcommand{\ltsname}{Diagramme de Transisions £\'E£tiquet£\'e£}
\end{lstlisting}
Crée une commande \lstinline?\ltsname?

qui écrit “\ltsname”.

\bigskip
Résultat :
\begin{columns}
\begin{column}{.45\textwidth}
\begin{lstlisting}
Soit $D$ un \ltsname.
\end{lstlisting}
\end{column}

\begin{column}{.45\textwidth}
\myex{%
Soit $D$ un \ltsname.
}
\end{column}
\end{columns}

\end{frame}



\begin{frame}[fragile]
  \frametitle{Définir des commandes avec arguments}

\begin{lstlisting}
\newcommand{\£\meta{nomcommande}£}[£\meta{n}£]{£\meta{Corps avec \#1, ..., \#n}£}
\end{lstlisting}
On peut définir un certain nombre d'arguments obligatoires pour une commande, et y faire référence avec \lstinline?#1?, \lstinline?#2?, ... jusqu'à \lstinline?#n?.

\bigskip
Exemple :
\begin{lstlisting}
\newcommand{\abs}[1]{\left|#1\right|}
\end{lstlisting}
Crée une commande \lstinline?\abs{xxx}?

qui permet de représenter la valeur absolue de son argument : “$\abs{xxx}$”.

\bigskip
Résultat :
\begin{columns}
\begin{column}{.45\textwidth}
\begin{lstlisting}
\begin{equation}
  \abs{\sum_{n=i}^j n} \leq
  \sum_{n=i}^j \abs{n}
\end{equation}
\end{lstlisting}
\end{column}

\begin{column}{.45\textwidth}
\myex{%
\begin{equation}
  \abs{\sum_{n=i}^j n} \leq
  \sum_{n=i}^j \abs{n}
\end{equation}
}
\end{column}
\end{columns}

\end{frame}



\begin{frame}[fragile]
  \frametitle{Définir des commandes avec un argument optionnel}

\begin{lstlisting}
\newcommand{\£\meta{nomcommande}£}[£\meta{n}£][£\meta{Valeur par d£\'e£faut}£]{£\meta{Corps avec \#1, ..., \#n}£}
\end{lstlisting}
On peut aussi définir une valeur par défaut pour le premier argument, qui sera utilisée quand cet argument n'est pas précisé.
Il faut alors écrire le premier argument entre crochets.

\bigskip
Exemple :
\begin{lstlisting}
\newcommand{\lts}[1][]{\left(Q^{#1},q_0^{#1},A_{#1},\rightarrow_{#1}\right)}
\end{lstlisting}
Crée une commande qui peut être utilisée avec un argument : \lstinline?\lts[xxx]?
ou sans argument : \lstinline?\lts?.

\bigskip
Résultat :
\begin{columns}
\begin{column}{.45\textwidth}
\begin{lstlisting}
\begin{equation}
  \lts[n]
\end{equation}
\begin{equation}
  \lts
\end{equation}
\end{lstlisting}
\end{column}

\begin{column}{.45\textwidth}
\myex{%
\begin{equation}
  \lts[n]
\end{equation}
\begin{equation}
  \lts
\end{equation}
}
\end{column}
\end{columns}

\end{frame}




\begin{frame}[fragile]
  \frametitle{Redéfinir des commandes}

\begin{lstlisting}
\renewcommand{\£\meta{nomcommande}£}[£\meta{n}£][£\meta{Valeur par d£\'e£faut}£]{£\meta{Corps avec \#1, ..., \#n}£}
\end{lstlisting}
On peut redéfinir une commande existante.

\bigskip
Exemple :
\begin{lstlisting}
\renewcommand{\vec}[1]{\overrightarrow{#1}}
\end{lstlisting}

\bigskip
Avant :
\begin{columns}
\begin{column}{.45\textwidth}
\begin{lstlisting}
\begin{equation}
  \vec{AB}
\end{equation}
\end{lstlisting}
\end{column}
\begin{column}{.45\textwidth}
\myex{%
\begin{equation}
  \vec{AB}
\end{equation}
}
\end{column}
\end{columns}

% À éviter ! Il vaut mieux (re)définir les commandes dans le préambule.
% Cette redéfinition ici est pour l'exemple.
\renewcommand{\vec}[1]{\overrightarrow{#1}}

\bigskip
Après :
\begin{columns}
\begin{column}{.45\textwidth}
\begin{lstlisting}
\begin{equation}
  \vec{AB}
\end{equation}
\end{lstlisting}
\end{column}
\begin{column}{.45\textwidth}
\myex{%
\begin{equation}
  \vec{AB}
\end{equation}
}
\end{column}
\end{columns}

\end{frame}



\section{Comprendre la compilation}

\begin{frame}
	\frametitle{Fichiers auxiliaires}
	\begin{description}
	\item[log] Fichier où \LaTeX{} écrit tout un tas d'informations sur la dernière compilation.
	\item[aux] Fichier auxiliaire: stocke les références, citations, numéros de page, etc.
	\item[toc] Fichier contenant la table des matières.
	\item[lof] Fichier contenant la liste des figures.
	\item[lot] Fichier contenant la liste des tables.
	\item[bbl] Fichier contenant la bibliographie.
	\end{description}
\end{frame}

\setbeamercovered{transparent}
\begin{frame}
\frametitle{Cycle de compilation}
\begin{center}
    \begin{tikzpicture}[every node/.style={shape=rectangle, shape aspect=1.61, rounded corners},
                        every path/.style={thick, >=triangle 60},
                        file/.style={minimum height=0.8cm, minimum width=1.29, fill=blue!20, draw=blue},
                        bin/.style={minimum width=2.42cm, minimum height=1.5cm, fill=red!20, draw=red}]
                        \only<1, 3-4>{\node[bin] (bin) at (0,0) {\LaTeX{}};}
                        \only<2>{\node[bin] (bin) at (0,0) {Bib\TeX{}};}
                        \onslide<1-4>{\node[file] (tex) at (-3, 1) {\texttt{.tex}};}
                        \onslide<1-4>{\node[file] (bib) at (-3, 0) {\texttt{.bib}};}
                        \onslide<1-4>{\node[file] (bst) at (-3, -1) {\texttt{.bst}};}

                        \onslide<3-4>{\node[file] (pdf) at (3,0.75) {\texttt{.pdf}};}
                        \onslide<1-4>{\node[file] (log) at (3, -0.75) {\texttt{.log}};}

                        \onslide<2-4>{\node[file] (aux) at (-1.5, 2) {\texttt{.aux}};}
                        \onslide<3-4>{\node[file] (bbl) at (0, 2) {\texttt{.bbl}};}
                        \onslide<3-4>{\node[file] (blg) at (1.5, 2) {\texttt{.blg}};}

                        \onslide<4>{\node[file] (toc) at (-1.5, -2) {\texttt{.toc}};}
                        \onslide<4>{\node[file] (lof) at (0, -2) {\texttt{.lof}};}
                        \onslide<4>{\node[file] (lot) at (1.5, -2) {\texttt{.lot}};}

                        \only<1>{
                            \draw[->] (tex) -- (bin);
                            \draw[->] (bin) -- (log);
                            \draw[->] (bin) -- (pdf);
                            \draw[->] (bin) -- (aux);
                        }
                        \only<2>{
                            \draw[->] (bib) -- (bin);
                            \draw[->] (aux) -- (bin);
                            \draw[->] (bin) -- (blg);
                            \draw[->] (bin) -- (bbl);
                        }
                        \only<3>{
                            \draw[->] (tex) -- (bin);
                            \draw[<->] (aux) -- (bin);
                            \draw[->] (bst) -- (bin);
                            \draw[->] (bbl) -- (bin);
                            \draw[->] (bin) -- (log);
                            \draw[->] (bin) -- (pdf);
                            \draw[->] (bin) -- (toc);
                            \draw[->] (bin) -- (lof);
                            \draw[->] (bin) -- (lot);
                        }
                        \only<4>{
                            \draw[->] (tex) -- (bin);
                            \draw[<->] (aux) -- (bin);
                            \draw[->] (bst) -- (bin);
                            \draw[->] (bbl) -- (bin);
                            \draw[->] (bin) -- (log);
                            \draw[->] (bin) -- (pdf);
                            \draw[<->] (bin) -- (toc);
                            \draw[<->] (bin) -- (lof);
                            \draw[<->] (bin) -- (lot);
                        }
    \end{tikzpicture}
\end{center}
\end{frame}

\begin{frame}[fragile]
  \frametitle{Erreurs de compilation courantes}

\begin{lstlisting}[language={},backgroundcolor=\color{gray!30}]
| ! Too many }'s.
| l.6 \date December 2004}
\end{lstlisting}
Accolades non balancées :
\begin{lstlisting}[backgroundcolor=\color{red!20}]
\date December 2004}
\end{lstlisting}\vspace*{-.85em}
\begin{lstlisting}[backgroundcolor=\color{green!20}]
\date{December 2004}
\end{lstlisting}

\pause
\hrule
\medskip
\begin{lstlisting}[language={},backgroundcolor=\color{gray!30}]
| ! Undefined control sequence.
| l.6 \dtae
| {December 2004}
\end{lstlisting}
Faute de frappe dans une commande, ou package manquant :
\begin{lstlisting}[backgroundcolor=\color{red!20}]
\dtae{December 2004}
\end{lstlisting}\vspace*{-.85em}
\begin{lstlisting}[backgroundcolor=\color{green!20}]
\date{December 2004}
\end{lstlisting}

\pause
\hrule
\medskip
\begin{lstlisting}[language={},backgroundcolor=\color{gray!30}]
| ! Missing $ inserted
\end{lstlisting}
Caractère math utilisé hors du mode math (ex: \_) :
\begin{lstlisting}[backgroundcolor=\color{red!20}]
Ouvrir le fichier ex_1.txt
\end{lstlisting}\vspace*{-.85em}
\begin{lstlisting}[backgroundcolor=\color{green!20}]
Ouvrir le fichier ex\_1.txt
\end{lstlisting}

\end{frame}

\setbeamercovered{invisible}
%%%%%%%%%%%%%%%%%%%%%%%%%%%%%%
%%%%%%%%%%% SECTION %%%%%%%%%%%%%%
%%%%%%%%%%%%%%%%%%%%%%%%%%%%%%

\section{Autres éditeurs \LaTeX}


%%%%%%%%% SLIDE %%%%%%%%%%%%%%%%%%

\begin{frame}{Texniccenter}

\begin{figure}
\centering
\includegraphics[width=9cm]{img/Texniccenter}
\end{figure}

{\footnotesize Téléchargement: \url{http://www.texniccenter.org/}}

\end{frame}


%%%%%%%%% SLIDE %%%%%%%%%%%%%%%%%%

\begin{frame}{LyX}

\begin{figure}
\centering
\includegraphics[width=8cm]{img/LyXScreen_Linux_en}
\end{figure}

{\footnotesize Téléchargement: \url{http://www.lyx.org/}}

\end{frame}


%%%%%%%%% SLIDE %%%%%%%%%%%%%%%%%%

\begin{frame}{Texmaker}

\begin{figure}
\centering
\includegraphics[width=9cm]{img/TexmakerView}
\end{figure}

{\footnotesize Téléchargement: \url{http://www.xm1math.net/texmaker/}}

\end{frame}



%%%%%%%%%%%%%%%%%%%%%%%%%%%%%%
%%%%%%%%%%% PART%% %%%%%%%%%%%%%%
%%%%%%%%%%%%%%%%%%%%%%%%%%%%%%

%%% TikZ
%%% Présentation de TikZ

\part{Inclusion de figures à l'aide de PGF/TikZ}

\section{Présentation de PGF/TikZ}

\begin{frame}
  \frametitle{PGF/TikZ : du dessin vectoriel en \LaTeX}

Qu'est-ce que PGF/TikZ ?
\begin{itemize}
  \item PGF est un langage \bemph{complet} et \bemph{compliqué} de dessin vectoriel,
  \item TikZ est une surcouche \bemph{plus simple} pour utiliser PGF.
\end{itemize}

\bigskip
Ils permettent de dessiner des figures facilement. Beaucoup d'avantages :
\begin{itemize}
  \item les figures sont \bemph{intégrés} au document \LaTeX{} (pas de fichier externe),
  \item dessin \bemph{vectoriel} : toujours lisse, quel que soit le niveau de zoom,
  \item très \bemph{riche}, beaucoup d'\bemph{exemples} disponibles faciles à reprendre.
\end{itemize}

\bigskip
Inconvénients :
\begin{itemize}
  \item parfois \bemph{difficile} à prendre en main,
  \item peut \bemph{alourdir} la compilation et le fichier final,
  \item ne permet pas de tout faire (mais presque).
\end{itemize}
\end{frame}



\section{Quelques exemples avec TikZ}

\begin{frame}
  \begin{figure}
    \centering
    \tikzexa
    \caption{\footnotesize Modèle d'architecture --- TEXample.net \cite{tikzandpgfexamples}}
  \end{figure}
\end{frame}

\begin{frame}
  \begin{figure}
    \centering
    \tikzexc
    \caption{\footnotesize Graphe simple --- TEXample.net \cite{tikzandpgfexamples}}
  \end{figure}
\end{frame}

\begin{frame}
  \begin{figure}
    \centering
    \scalebox{0.7}{\tikzexe}
    \caption{\footnotesize Circuit électrique --- TEXample.net \cite{tikzandpgfexamples}}
  \end{figure}
\end{frame}

\begin{frame}
  \begin{figure}
    \centering
    \scalebox{0.7}{\tikzexd}
    \caption{\footnotesize Incidence oblique --- TEXample.net \cite{tikzandpgfexamples}}
  \end{figure}
\end{frame}

\begin{frame}
  \begin{figure}
    \centering
    \scalebox{0.7}{\tikzexb}
    \caption{\footnotesize Microscope électronique à transmission --- TEXample.net \cite{tikzandpgfexamples}}
  \end{figure}
\end{frame}





\section{Utilisation de TikZ}

\begin{frame}[fragile]
  \frametitle{Préambule}

TikZ doit être chargé dans le préambule :
\lstinline?\usepackage{tikz}?

\medskip
On peut aussi charger des bibliothèques propres à TikZ dans le préambule avec :
\lstinline?\usetikzlibrary{bibliotheques}?,
ce qui permet d'utiliser :
\begin{itemize}
  \item de nouvelles formes de pointes de flèches (\lstinline?arrows?),
  \item des dégradés (\lstinline?shadings?),
  \item des styles de lignes (\lstinline?decorations.pathmorphing?),
  \item etc.
\end{itemize}
\end{frame}



\begin{frame}[fragile]
  \frametitle{Création d'une figure}

\bigskip
Dans le document, on définit une image TikZ à l'aide de l'environnement \lstinline?tikzpicture?,
souvent inclus dans une \lstinline?figure? :

\begin{lstlisting}
\begin{figure}
  \begin{tikzpicture}
    ...
    ...    % Contenu de l'image
    ...
  \end{tikzpicture}
  \caption{...}
  \label{...}
\end{figure}
\end{lstlisting}
\end{frame}



% Morceaux de la figure d'exemple (pour éviter les copier-coller)
\def \tikzexnodes
{
\node[circle, fill=yellow, draw] (rond) {1};
\node[ellipse, fill=red!50, right of=rond, node distance=3cm]
  (ellipse) {Une ellipse};
\node[diamond, fill=blue!50, draw=blue, thick] at (-2, 0)
  (diamantvide) {};
}

\def \tikzexedgesa
{
\path[->] (rond) edge (ellipse);
\path[o->>, bend right, dashed] (rond) edge (diamantvide);
}

\def \tikzexedgesb
{
\path[o->>, bend right] (diamantvide) edge
  node[below, fill=green!30] (retour) {retour}
  (rond);
\path[<->, bend right] (retour.east) edge (rond.south);
}

\def \tikzexedges
{
\tikzexedgesa
\tikzexedgesb
}

\begin{frame}[fragile,t]
  \frametitle{Description de l'image avec TikZ}

\begin{figure}
  \scalebox{0.6}{\tikzexc}
\end{figure}

Une figure TikZ est constituée d'éléments définis à l'aide de commandes :

\begin{lstlisting}
  \commande[param£\`e£tres] ... suite de la c££ommande ... ;
\end{lstlisting}

Par exemple, un graphe est composé de nœuds et d'arcs entre ces nœuds.
Tous sont définis à l'aide de commandes TikZ \lstinline?\node? et \lstinline?\path?.
\end{frame}



\begin{frame}[fragile, t]
  \frametitle{Exemple : un graphe simple}

\begin{figure}
  \begin{tikzpicture}
    \tikzexnodes
  \end{tikzpicture}
\end{figure}

On définit un nœud avec la commande \lstinline?\node? :
\begin{lstlisting}
\node[£\meta{options}£] (£\meta{nom}£) {£\meta{\'etiquette}£};
\end{lstlisting}

On peut spécifier :
\medskip
\begin{itemize}
  \item le nom interne \lstinline?(nom)?,
  \item l'étiquette visible \lstinline?{etiquette}?,
  \item la forme (\lstinline?circle?, \lstinline?ellipse?, \lstinline?square?, \lstinline?diamond?),
  le type de ligne et la couleur de fond,
  la position (absolue ou par rapport aux autres nœuds), ...
\end{itemize}

% Note : « n££ode » permet uniquement d'écrire « node » sans le colorer das le listing
\begin{lstlisting}
\node[circle, fill=yellow, draw] (rond) {1};
\node[ellipse, fill=red!50, right of=rond, node distance=3cm]
  (ellipse) {Une ellipse};
\node[diamond, fill=blue!50, draw=blue, thick] at (-2, 0) (diamantvide) {};
\end{lstlisting}
\end{frame}



\begin{frame}[fragile, t]
  \frametitle{Exemple : un graphe simple}

\begin{figure}
  \begin{tikzpicture}
    \tikzexnodes
    \tikzexedgesa
  \end{tikzpicture}
\end{figure}

On définit ensuite un arc entre deux nœuds avec la commande

\begin{lstlisting}
    \path[£\meta{options}£] (£\meta{origine}£) edge (£\meta{cible}£);
\end{lstlisting}

On peut définir :
\begin{itemize}
  \item l'\lstinline?(origine)? et la \lstinline?(cible)? grâce à leurs noms internes,
  \item le type de flèche (\lstinline?->?, \lstinline?o->?, \lstinline?-?),
    la courbure (\lstinline?bend right?),
    le type de trait (\lstinline?thick?, \lstinline?dashed?), ...
\end{itemize}

\begin{lstlisting}
\path[->] (rond) edge (ellipse);
\path[o->>, bend right, dashed] (rond) edge (diamantvide);
\end{lstlisting}

\end{frame}



\begin{frame}[fragile, t]
  \frametitle{Exemple : un graphe simple}

\begin{figure}
  \begin{tikzpicture}
    \tikzexnodes
    \tikzexedges
  \end{tikzpicture}
\end{figure}

On peut placer un nouveau nœud sur un arc avec le mot-clef \lstinline?node? :

\begin{lstlisting}
\path[o->>, bend right] (diamantvide) edge
  node[below, fill=green!30] (retour) {retour}
  (rond);
\end{lstlisting}

Il se comporte comme un nœud normal (on peut y faire référence normalement).

\medskip
On peut aussi définir d'où partent les arcs :

\begin{lstlisting}
\path[<->, bend right] (retour.east) edge (rond.south);
\end{lstlisting}

\end{frame}



\section{Conclusion sur TikZ}

\begin{frame}
  \frametitle{Réutiliser au maximum}

Pour produire de belles figures TikZ, le mieux est de chercher des exemples et de les modifier.

\begin{center}
Pour cela : \Huge \rotatebox[origin=c]{15}{\myheart}\bemph{Internet !}\rotatebox[origin=c]{-15}{\myheart}
\end{center}

On pourra notamment se servir des exemples disponibles sur TEXample \cite{tikzandpgfexamples} à : \url{http://texample.net/tikz/examples/}.

\bigskip
De plus, il est possible :
\begin{itemize}
  \item de définir des thèmes pour des figures semblables,
  \item d'utiliser des bibliothèques pour des diagrammes répandus (UML, schémas électriques...).
\end{itemize}
\end{frame}


%%% Beamer
%%% Présentation de Beamer

\part{Beamer}

\section{Utilisation de Beamer}

\begin{frame}[fragile]
  \frametitle{Qu'est-ce que Beamer ?}

Beamer est une classe \LaTeX{} :

\lstinline?\documentclass{beamer}?

\medskip
Points communs :
\begin{itemize}
  \item structuration (parties, sections, sous-sections ; pas de chapitres),
  \item mise en forme du texte,
  \item inclusion de figures et de formules mathématiques,
  \item etc.
\end{itemize}

\medskip
Différences :
\begin{itemize}
  \item structuration en diapositives,
  \item nouvelles commandes (transitions),
  \item mise en page différente (police, agencement).
\end{itemize}
\end{frame}



\begin{frame}[fragile]
  \frametitle{Définition du document}

Beamer est une classe \LaTeX{} :

\lstinline?\documentclass[options]{beamer}?

\medskip
Parmi les \lstinline?options? :
\begin{itemize}
  \item \lstinline?t?, \lstinline?c? ou \lstinline?b? pour aligner verticalement le texte en haut, au milieu ou en bas de la diapositive,
  \item \lstinline?Xpt? pour définir la taille de la police à \lstinline?X? (ex : \lstinline?9pt?),
  \item \lstinline?handout? pour obtenir une version imprimable (sans transitions/animations).
\end{itemize}

\medskip
Puis le préambule, et le contenu du document dans :
\begin{lstlisting}
\begin{document}
  ...
  ...    % Les diapositives ici
  ...
\end{document}
\end{lstlisting}
\end{frame}



\begin{frame}[fragile]
  \frametitle{Définition d'une diapositive}

Chaque diapositive est comprise dans un environnement \lstinline?frame? :

\begin{lstlisting}
 \begin{frame}[options]
   ...
   ...    % Contenu de la diapositive
   ...
 \end{frame}
\end{lstlisting}

\medskip
Les \lstinline?options? peuvent contenir :
\begin{itemize}
  \item \lstinline?t?, \lstinline?c? ou \lstinline?b? pour changer l'alignement vertical du texte pour cette diapositive uniquement,
  \item \lstinline?plain? pour ne pas afficher les bandeaux d'en-tête et de pied pour cette diapositive,
  \item \lstinline?shrink? pour tasser le texte s'il y en a beaucoup,
  \item \lstinline?fragile? si la diapositive contient du code (comme ici).
\end{itemize}
\end{frame}



\begin{frame}[fragile]
  \frametitle{Propriétés d'une diapositive}
  \framesubtitle{Titre, sous-titre et bandeaux}

On peut définir un titre et un sous-titre pour une diapositive :

\begin{lstlisting}
\frametitle{£\meta{Titre de la diapo}£}
\framesubtitle{£\meta{Sous-titre de la diapo}£}
\end{lstlisting}

\bigskip
De plus, des informations relatives au thème s'affichent dans les bandeaux d'en-tête et de pied :
\begin{itemize}
  \item section en cours,
  \item titre de la présentation, date, nom des auteurs et institut,
  \item numérotation des diapositives.
\end{itemize}
\end{frame}



\begin{frame}[fragile]
  \frametitle{À l'intérieur d'une diapositive}

Le contenu d'une diapositive est du \LaTeX{} habituel :
\begin{itemize}
  \item listes,
  \item figures (contenant tableaux, figures complexes, images...),
  \item texte et équations mathématiques,
  \item etc.
\end{itemize}

\medskip
On peut aussi englober ces éléments dans des blocs :
\begin{lstlisting}
\begin{exampleblock}{Titre du bloc}
  Contenu du bloc (listes, £\'e£quations, maths, ...)
\end{exampleblock}
\end{lstlisting}

\begin{exampleblock}{Titre du bloc}
  Contenu du bloc (listes, équations, maths, ...)
\end{exampleblock}

\medskip
3 types de blocs : \lstinline?block?, \lstinline?alertblock? et \lstinline?exampleblock?.
\end{frame}



\begin{frame}[plain]
\begin{figure}
  \centering
  \includegraphics[width=1\textwidth]{img/seance3_extheme_madrid}
\end{figure}
\end{frame}



\section{Les animations en Beamer}

\begin{frame}[fragile]
  \frametitle{Animations}

On peut définir des animations (statiques) au sein des présentations.
\begin{itemize}
  \item<2,4-> Elles consistent en des apparitions...
  \item<3-> ...ou des disparitions.
\end{itemize}

\bigskip
\pause[4]
Les animations créent plusieurs pages pour la même diapositive, avec les différences nécessaires. La numérotation n'est pas affectée.

\medskip
L'option \lstinline?handout? du \lstinline?\documentclass? permet de supprimer ou de simplifier ces animations.
\end{frame}



\begin{frame}[fragile]
  \frametitle{Apparitions successives}

Avec la commande \lstinline?\pause? ou \lstinline?\pause[x]?

\medskip
Exemple :

\begin{columns}
  \begin{column}{0.20\textwidth}
\begin{lstlisting}
| Texte 1
| \pause
| Texte 2
| 
| \pause
| Texte 3
| \pause
| Texte 4
| \pause[3]
| Texte 5
\end{lstlisting}
  \end{column}
  \begin{column}{0.35\textwidth}
\rmfamily
Texte 1
\pause{}
Texte 2

\pause
Texte 3
\pause{}
Texte 4
\pause[3]
Texte 5
  \end{column}
\end{columns}
\end{frame}



\begin{frame}[fragile]
  \frametitle{Animations avancées}

Deux commandes :
\begin{itemize}
  \item \lstinline?\only<pages>{contenu}? dévoile \lstinline?contenu? uniquement dans les \lstinline?pages? spécifiées,
  \item \lstinline?\uncover<pages>{contenu}? fait de même, mais réserve l'espace non occupé lorsqu'il n'est pas affiché.
\end{itemize}

Le \lstinline?contenu? peut contenir n'importe quoi (texte, figures, mathématiques, etc.).

Les \lstinline?<pages>? sont définies par groupes :
\begin{itemize}
  \item \lstinline?<n>? : la page $n$,
  \item \lstinline?<-n>? : toutes les pages avant $n$ compris,
  \item \lstinline?<n->? : toutes les pages à partir de $n$,
  \item \lstinline?<n-p>? : toutes les pages entre $n$ et $p$ inclus,
  \item \lstinline?<x,y>? : le groupe de pages $x$ et le groupe de pages $y$.
\end{itemize}
\end{frame}



\begin{frame}[fragile]
  \frametitle{Animations avancées}

Exemple avec \lstinline?\only? et \lstinline?\uncover? :
\begin{columns}
  \begin{column}{0.35\textwidth}
\begin{lstlisting}
| Texte 1.
| 
| \uncover<2->{
|   Texte 2 ?
|   \only<2-4>{Texte 3...}
|   \uncover<3>{Texte 4 !}
|   Texte 5.
| }
| \pause[5]
\end{lstlisting}
  \end{column}
  \begin{column}{0.60\textwidth}
\rmfamily
Texte 1.

\uncover<2->{
  Texte 2 ?
  \only<2-4>{Texte 3...}
  \uncover<3>{Texte 4 !}
  Texte 5.
}
\pause[5]
  \end{column}
\end{columns}
\end{frame}



\begin{frame}[fragile]
  \frametitle{Animations avancées}

D'autres commandes peuvent prendre un argument \lstinline?<pages>? optionnel.

\bigskip
Exemple : \lstinline?\item<pages>?
\begin{columns}
  \begin{column}{0.40\textwidth}
\begin{lstlisting}
\begin{itemize}
  \item<1,5> Premier £\'e£l£\'e£ment
  \item<2,4-> Second £\'e£l£\'e£ment
  \item<3-> Troisi£\`e£me £\'e£l£\'e£ment
\end{itemize}\end{lstlisting}
  \end{column}
  \begin{column}{0.40\textwidth}
\rmfamily
\begin{itemize}
  \item<1,5> Premier élément
  \item<2,4-> Second élément
  \item<3-> Troisième élément
\end{itemize}
  \end{column}
\end{columns}

\pause[5]
\end{frame}



\begin{frame}[fragile, t]
  \frametitle{Animations TikZ}

\begin{figure}
  \begin{tikzpicture}
    \tikzexnodes
    \tikzexedges
\node<2> at (rond) [rectangle, fill=green!20, draw, thick] {Oui !} ;
\node<3> at (ellipse) [rectangle, fill=red!20, draw, thick] {Non !} ;
  \end{tikzpicture}
\end{figure}

\bigskip
Beaucoup de commandes TikZ acceptent aussi la syntaxe \lstinline?<pages>? pour créer des animations dans une présentation.

\bigskip
\begin{lstlisting}
\node<2> at (rond) [square, fill=green!20, draw, thick] {Oui !} ;
\node<3> at (ellipse) [square, fill=red!20, draw, thick] {Non !} ;
\end{lstlisting}
\end{frame}



\section{Personnalisation de Beamer}

\begin{frame}[fragile]
  \frametitle{Les thèmes}

Il est possible d'utiliser des thèmes prédéfinis pour modifier l'apparence et les couleurs d'une présentation. On peut spécifier :
\begin{itemize}
  \item Un thème d'agencement avec \lstinline?\usetheme{theme}? :
  \begin{itemize}
    \item style de la page de titre et agencement des diapos,
    \item forme et contenu des bandeaux,
    \item police, forme des puces, ...
  \end{itemize}
  Exemples : \lstinline?Warsaw?, \lstinline?Madrid?, \lstinline?Copenhagen?, \lstinline?CambridgeUS?...
  \item Un thème de couleurs avec \lstinline?\usecolortheme{theme}? :
  \begin{itemize}
    \item couleur du texte, des titres, du sommaire,
    \item couleur de fond, des blocs, des bandeaux...
  \end{itemize}
  Exemples : \lstinline?beaver?, \lstinline?dolphin?, \lstinline?dove?, \lstinline?fly?...
\end{itemize}

\bigskip
Pour une liste des thèmes par défaut, voir le WikiBooks \cite{wikibooksbeamer}.
\end{frame}



\begin{frame}[fragile]
  \frametitle{Personnaliser un thème}

Il est aussi possible de personnaliser en partie un thème ou de créer un thème, pour :
\begin{itemize}
  \item modifier le contenu des bandeaux d'en-tête et de pied,
  \item revoir l'agencement,
  \item supprimer des éléments inutiles (sommaire, icônes...),
  \item adapter certaines couleurs.
\end{itemize}

\bigskip
On peut pour cela redéfinir toutes les caractéristiques d'une présentation :
\begin{itemize}
  \item les agencements,
  \item les couleurs.
\end{itemize}

\bigskip
Pour une liste des options modifiables, voir le WikiBooks \cite{wikibooksbeamer}.
\end{frame}



\begin{frame}[fragile]
  \frametitle{Exemple de thème : CambridgeUS}

\begin{block}{Bloc normal (neutre)}
  Contenu du bloc (listes, équations, maths, ...)
\end{block}

\begin{alertblock}{Bloc d'alerte}
  Si on suppose :
  \begin{equation}
    1+1=0
  \end{equation}
  alors on peut prouver n'importe quoi.
\end{alertblock}

\begin{exampleblock}{Bloc d'exemple}
  Par exemple :
  \begin{itemize}
    \item Tout ce qui est vrai est aussi faux, et inversement,
    \item $x = y$ pour tout $x$ et tout $y$,
    \item mon chat et moi ne formons qu'une seule personne.
  \end{itemize}
\end{exampleblock}
\end{frame}



\begin{frame}[plain]
\begin{figure}
  \centering
  \includegraphics[width=1\textwidth]{img/seance3_extheme_madrid}
\end{figure}
\end{frame}



\begin{frame}[plain]
\begin{figure}
  \centering
  \includegraphics[width=1\textwidth]{img/seance3_extheme_ecn}
\end{figure}
\end{frame}



\begin{frame}[fragile]
  \frametitle{Exercice}

Une présentation simple :

\begin{lstlisting}[multicols=2]
  \documentclass{beamer}

  \usepackage[french]{babel}
  \usepackage[utf8]{inputenc}

  \usetheme{Madrid}
  \usecolortheme{default}

  \title{Pr£\'e£sentation de ma th£\`e£se}
  \author{Pr£\'e£nom Nom}
  \institute[LDC]{Laboratoire des Chatons}

  \begin{document}

  \begin{frame}
    \maketitle
  \end{frame}

  
  \section{£\`A£ propos de moi}

  \begin{frame}
    \frametitle{Ce que j'aime}
    \begin{itemize}
      \item Les chatons,
      \pause
      \item le jus de raisin,
      \pause
      \item etc.
    \end{itemize}
  \end{frame}

  \end{document}
\end{lstlisting}

\end{frame}






%%% Bibliographie %%%

\begin{frame}
  \frametitle{Bibliographie}

\nocite{*}
\bibliographystyle{abbrv-fr}
\bibliography{latex}
\end{frame}


%%%%%%%%%%%%%%%%%%%%%%%%%%%%%%%%%%%%%%%%%%%%%%%%%%%%%%%%%%%%%%%%%%%%%%%%%%%%%%%%%%
%%%%%%%%%%%%%%%%%%%%%%%% FIN DU DOCUMENT %%%%%%%%%%%%%%%%%%%%%%%%%%%%%%%%%%%%%%%%%%%%%%%%
%%%%%%%%%%%%%%%%%%% ( NON PRISE EN COMPTE DE LA SUITE ) %%%%%%%%%%%%%%%%%%%%%%%%%%%%%%%%%%%%%%%%%%%
\end{document}
%%%%%%%%%%%%%%%%%%%%%%%%%%%%%%%%%%%%%%%%%%%%%%%%%%%%%%%%%%%%%%%%%%%%%%%%%%%%%%%%%%
%%%%%%%%%%%%%%%%%%%%%%%%%%%%%%%%%%%%%%%%%%%%%%%%%%%%%%%%%%%%%%%%%%%%%%%%%%%%%%%%%%
