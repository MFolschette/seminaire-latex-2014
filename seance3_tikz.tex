%%% Présentation de TikZ

\part{Inclusion de figures à l'aide de PGF/TikZ}

\section{Présentation de PGF/TikZ}

\begin{frame}
  \frametitle{PGF/TikZ : du dessin vectoriel en \LaTeX}

Qu'est-ce que PGF/TikZ ?
\begin{itemize}
  \item PGF est un langage \bemph{complet} et \bemph{compliqué} de dessin vectoriel,
  \item TikZ est une surcouche \bemph{plus simple} pour utiliser PGF.
\end{itemize}

\bigskip
Ils permettent de dessiner des figures facilement. Beaucoup d'avantages :
\begin{itemize}
  \item les figures sont \bemph{intégrés} au document \LaTeX{} (pas de fichier externe),
  \item dessin \bemph{vectoriel} : toujours lisse, quel que soit le niveau de zoom,
  \item très \bemph{riche}, beaucoup d'\bemph{exemples} disponibles faciles à reprendre.
\end{itemize}

\bigskip
Inconvénients :
\begin{itemize}
  \item parfois \bemph{difficile} à prendre en main,
  \item peut \bemph{alourdir} la compilation et le fichier final,
  \item ne permet pas de tout faire (mais presque).
\end{itemize}
\end{frame}



\section{Quelques exemples avec TikZ}

\begin{frame}
  \begin{figure}
    \centering
    \tikzexa
    \caption{\footnotesize Modèle d'architecture --- TEXample.net \cite{tikzandpgfexamples}}
  \end{figure}
\end{frame}

\begin{frame}
  \begin{figure}
    \centering
    \tikzexc
    \caption{\footnotesize Graphe simple --- TEXample.net \cite{tikzandpgfexamples}}
  \end{figure}
\end{frame}

\begin{frame}
  \begin{figure}
    \centering
    \scalebox{0.7}{\tikzexe}
    \caption{\footnotesize Circuit électrique --- TEXample.net \cite{tikzandpgfexamples}}
  \end{figure}
\end{frame}

\begin{frame}
  \begin{figure}
    \centering
    \scalebox{0.7}{\tikzexd}
    \caption{\footnotesize Incidence oblique --- TEXample.net \cite{tikzandpgfexamples}}
  \end{figure}
\end{frame}

\begin{frame}
  \begin{figure}
    \centering
    \scalebox{0.7}{\tikzexb}
    \caption{\footnotesize Microscope électronique à transmission --- TEXample.net \cite{tikzandpgfexamples}}
  \end{figure}
\end{frame}





\section{Utilisation de TikZ}

\begin{frame}[fragile]
  \frametitle{Préambule}

TikZ doit être chargé dans le préambule :
\lstinline?\usepackage{tikz}?

\medskip
On peut aussi charger des bibliothèques propres à TikZ dans le préambule avec :
\lstinline?\usetikzlibrary{bibliotheques}?,
ce qui permet d'utiliser :
\begin{itemize}
  \item de nouvelles formes de pointes de flèches (\lstinline?arrows?),
  \item des dégradés (\lstinline?shadings?),
  \item des styles de lignes (\lstinline?decorations.pathmorphing?),
  \item etc.
\end{itemize}
\end{frame}



\begin{frame}[fragile]
  \frametitle{Création d'une figure}

\bigskip
Dans le document, on définit une image TikZ à l'aide de l'environnement \lstinline?tikzpicture?,
souvent inclus dans une \lstinline?figure? :

\begin{lstlisting}
\begin{figure}
  \begin{tikzpicture}
    ...
    ...    % Contenu de l'image
    ...
  \end{tikzpicture}
  \caption{...}
  \label{...}
\end{figure}
\end{lstlisting}
\end{frame}



% Morceaux de la figure d'exemple (pour éviter les copier-coller)
\def \tikzexnodes
{
\node[circle, fill=yellow, draw] (rond) {1};
\node[ellipse, fill=red!50, right of=rond, node distance=3cm]
  (ellipse) {Une ellipse};
\node[diamond, fill=blue!50, draw=blue, thick] at (-2, 0)
  (diamantvide) {};
}

\def \tikzexedgesa
{
\path[->] (rond) edge (ellipse);
\path[o->>, bend right, dashed] (rond) edge (diamantvide);
}

\def \tikzexedgesb
{
\path[o->>, bend right] (diamantvide) edge
  node[below, fill=green!30] (retour) {retour}
  (rond);
\path[<->, bend right] (retour.east) edge (rond.south);
}

\def \tikzexedges
{
\tikzexedgesa
\tikzexedgesb
}

\begin{frame}[fragile,t]
  \frametitle{Description de l'image avec TikZ}

\begin{figure}
  \scalebox{0.6}{\tikzexc}
\end{figure}

Une figure TikZ est constituée d'éléments définis à l'aide de commandes :

\begin{lstlisting}
  \commande[param£\`e£tres] ... suite de la c££ommande ... ;
\end{lstlisting}

Par exemple, un graphe est composé de nœuds et d'arcs entre ces nœuds.
Tous sont définis à l'aide de commandes TikZ \lstinline?\node? et \lstinline?\path?.
\end{frame}



\begin{frame}[fragile, t]
  \frametitle{Exemple : un graphe simple}

\begin{figure}
  \begin{tikzpicture}
    \tikzexnodes
  \end{tikzpicture}
\end{figure}

On définit un nœud avec la commande \lstinline?\node? :
\begin{lstlisting}
\node[£\meta{options}£] (£\meta{nom}£) {£\meta{\'etiquette}£};
\end{lstlisting}

On peut spécifier :
\medskip
\begin{itemize}
  \item le nom interne \lstinline?(nom)?,
  \item l'étiquette visible \lstinline?{etiquette}?,
  \item la forme (\lstinline?circle?, \lstinline?ellipse?, \lstinline?square?, \lstinline?diamond?),
  le type de ligne et la couleur de fond,
  la position (absolue ou par rapport aux autres nœuds), ...
\end{itemize}

% Note : « n££ode » permet uniquement d'écrire « node » sans le colorer das le listing
\begin{lstlisting}
\node[circle, fill=yellow, draw] (rond) {1};
\node[ellipse, fill=red!50, right of=rond, node distance=3cm]
  (ellipse) {Une ellipse};
\node[diamond, fill=blue!50, draw=blue, thick] at (-2, 0) (diamantvide) {};
\end{lstlisting}
\end{frame}



\begin{frame}[fragile, t]
  \frametitle{Exemple : un graphe simple}

\begin{figure}
  \begin{tikzpicture}
    \tikzexnodes
    \tikzexedgesa
  \end{tikzpicture}
\end{figure}

On définit ensuite un arc entre deux nœuds avec la commande

\begin{lstlisting}
    \path[£\meta{options}£] (£\meta{origine}£) edge (£\meta{cible}£);
\end{lstlisting}

On peut définir :
\begin{itemize}
  \item l'\lstinline?(origine)? et la \lstinline?(cible)? grâce à leurs noms internes,
  \item le type de flèche (\lstinline?->?, \lstinline?o->?, \lstinline?-?),
    la courbure (\lstinline?bend right?),
    le type de trait (\lstinline?thick?, \lstinline?dashed?), ...
\end{itemize}

\begin{lstlisting}
\path[->] (rond) edge (ellipse);
\path[o->>, bend right, dashed] (rond) edge (diamantvide);
\end{lstlisting}

\end{frame}



\begin{frame}[fragile, t]
  \frametitle{Exemple : un graphe simple}

\begin{figure}
  \begin{tikzpicture}
    \tikzexnodes
    \tikzexedges
  \end{tikzpicture}
\end{figure}

On peut placer un nouveau nœud sur un arc avec le mot-clef \lstinline?node? :

\begin{lstlisting}
\path[o->>, bend right] (diamantvide) edge
  node[below, fill=green!30] (retour) {retour}
  (rond);
\end{lstlisting}

Il se comporte comme un nœud normal (on peut y faire référence normalement).

\medskip
On peut aussi définir d'où partent les arcs :

\begin{lstlisting}
\path[<->, bend right] (retour.east) edge (rond.south);
\end{lstlisting}

\end{frame}



\section{Conclusion sur TikZ}

\begin{frame}
  \frametitle{Réutiliser au maximum}

Pour produire de belles figures TikZ, le mieux est de chercher des exemples et de les modifier.

\begin{center}
Pour cela : \Huge \rotatebox[origin=c]{15}{\myheart}\bemph{Internet !}\rotatebox[origin=c]{-15}{\myheart}
\end{center}

On pourra notamment se servir des exemples disponibles sur TEXample \cite{tikzandpgfexamples} à : \url{http://texample.net/tikz/examples/}.

\bigskip
De plus, il est possible :
\begin{itemize}
  \item de définir des thèmes pour des figures semblables,
  \item d'utiliser des bibliothèques pour des diagrammes répandus (UML, schémas électriques...).
\end{itemize}
\end{frame}
